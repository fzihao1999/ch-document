\documentclass[UTF8]{ctexart}
\usepackage{listings}
\usepackage{graphicx}
\usepackage{geometry}
\usepackage{amsmath}
\usepackage{palatino}
\usepackage{tikz}
\usepackage{xcolor}
\usepackage{amsthm,amsmath,amssymb}
\usepackage{mathrsfs}
\pagestyle{plain} %设置页眉页脚
\lstset{
    numbers=left, 
    numberstyle= \tiny, 
    keywordstyle= \color{ blue!70},
    commentstyle= \color{red!50!green!50!blue!50}, 
    frame=shadowbox, % 阴影效果
    rulesepcolor= \color{ red!20!green!20!blue!20} ,
    escapeinside=``, % 英文分号中可写入中文
    xleftmargin=2em,xrightmargin=2em, aboveskip=1em,
    framexleftmargin=2em
} 
\usetikzlibrary{shapes.geometric, arrows}
\geometry{left=2.5cm, right=2.5cm, top=2.5cm, bottom=2.5cm}
\numberwithin{equation}{section}
\begin{document}

\section{Related Work}
\subsection{Chinese Spelling Error Correction}
该文中提出了一种spelling error correction的方式,可以对于文章进行字符级别的查错与纠正。该文章的目标是将输入语句$X=(x_{1},x_{2},...,x_{n})$进行纠错之后输出为$Y=(y_{1},y_{2},...y_{n})$,其中$x_{i}$为语句中的第$i$个字符,$y_{i}$为$x_{i}$对应的修改后的字符。该系统主要由Detection Network和Correction Network两个部分构成,其中的Detection Network对于本项工作有一些帮助。

Detection Network以sequence of embedding  $E=(e_{1},e_{2},...,e_{n})$作为输入,其中$e_{i}$是$x_{i}$的embedding,更具体的说,是word embedding,position embedding 和 segment embedding的和。该模型输出$G=(g_{1},g_{2},...,g{n})$,其中$g_{i}$表示第$i$个字符是否正确,0表示字符正确,1表示错误。按照上述的定义,detection network采用了Bi-GRU,定义每个字符错误的概率$p_{i}$
\begin{equation}
p_{i}=P_{d}(g_{i}=1|X)=\sigma(W_{d}h_{i}^{d}+b_{d})
\end{equation}

$h_{i}^{d}$是Bi-GRU的hidden state,定义为:
\begin{equation}
\overrightarrow{h}_{i}^{d}=GRU(\overrightarrow{h}_{i-1}^{d},e_{i})
\end{equation}
\begin{equation}
\overleftarrow{h}_{i}^{d}=GRU(\overleftarrow{h}_{i+1}^{d},e_{i})
\end{equation}
\begin{equation}
h_{i}^{d}=\left [\overrightarrow{h}_{i}^{d},  \overleftarrow{h}_{i}^{d} \right ]
\end{equation}

将embedding 和 soft embedding的加权和作为soft-mask embedding:

\begin{equation}
e_{i}^{'}=p{i}*e_{mask}+(1-p_{i})*e_{i}
\end{equation}

该模型利用已经标注好的数据进行训练,即$(X_{i},Y_{i})$这样的数据对。对于我们的工作,不规范语句的标注而言,这样的训练对比较难以获得,因为规范的定义比较抽象,我们对于是否符合规范的语句不好进行定义。但是这篇文章的工作提供了一个规范化的一个方面,也是比较重要的一个方面。

可以尝试对定义规范化的一些规则,例如人称问题、seplling问题等。对于逐条规则进行check,然后对于不符合规则的进行highlight。

\end{document}